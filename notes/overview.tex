\documentclass{scrartcl}

\usepackage{amsmath}
\usepackage{amsfonts}
\usepackage{amsthm}
\usepackage{hyperref}
\usepackage{tikz}
\usepackage{xspace}
\usepackage{wrapfig}
\usepackage{subcaption}
\usepackage{nicefrac}
\usepackage{natbib}
\bibliographystyle{unsrtnat}
\usepackage[format=plain]{caption}
\usepackage{enumitem}

\setlength{\parindent}{0pt}
\setlength{\parskip}{2.5pt}
\setlist[enumerate]{leftmargin=*}

\title{\LARGE DeepMAsED: Evaluating the quality of metagenomic assembly}

\author{\large Mateo Rojas-Carulla
}

\date{}

\begin{document}
\maketitle

The goal of this project is to evaluate the quality of metagenomic assemblers from metagenomes. 
Thousands of genomes are being assembled to metagenomes~\citep{pasolli2019extensive}, but it remains unclear 
how good these assemblies actually are. Indeed, we lack a ground truth to compare against. We aim to develop a 
deep learning tool assigning a score to an assembled genome.

\section{Training protocol}

This issue is particularly well suited for a machine learning procedure since we can simulate data similar to real data. Indeed, 
the assembly of new genomes is as follows:
\begin{itemize}
  \item Sequence the DNA from a biological sample. This results in a large number of \emph{reads}, short sequences of DNA belonging 
    to the organisms present in the sample. 
  \item Select a Next Generation Sequencing (NGS) assembler, such as MEGAHIT~\citep{li2015megahit} or MetaSPAdes~\citep{nurk2017metaspades}. 
  \item Process the reads and output the genomes assembled by the chosen assembler. 
\end{itemize}

We can simulate large amounts of data by replacing the first step in this pipeline by generating reads synthetically from known genomes. 
The resulting, reassembled genomes can then be compared to the corresponding ground truth in order to assess the quality of the 
process. 

\section{Experiments}
\subsection{Small scale}
We start from a dataset containing 
\begin{itemize}
  \item $40$ genomes obtained from NCBI. 
  \item Reads generated using MGSIM at $10^7$ depth. 
  \item Contigs assembled with MetaSPAdes and MEGAHIT. 
  \item We consider three metrics. First, we consider a binary variable indicating whether the contig is chimeric. Second, we 
    consider a continuous variable, the edit distance. Third, we consider the ``extensive missassembly'' label provided by 
    metaQUAST. 
\end{itemize}

This results roughly in datasets with $12,000$ contigs. 



\bibliography{biblio}

\end{document}
